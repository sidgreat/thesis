The technique that we propose preserves sharpness of shadows and also models
specular reflection. This causes image to appear more photorealistic and single
point source effect is more prominent. This technique results in enhanced
photorealism which preserves sharp shadows and specular properties from
smoothening out. The separate model for luminance estimation provides us with
color values which are in close agreement with the color values of the original
image. Results obtained on re-rendering the input images show a great
improvement over original PTM technique.

% The principle advantage of image-based rendering techniques is that photorealism can be achieved without
% accurate modeling of complex real-world physical interactions. You can capture and reproduce interesting
% lighting effects directly from how they appear in reality. Since they are captured in photographs, complex
% interactions like self-shadowing, interreflections, and sub-surface scattering can be reproduced automatically. A
% related advantage is that the rendering cost is independent of the complexity of the scene and the surface 
% properties of objects in the scene. The complexity of image-based methods usually depends on the number of 
% images or on the representation rather than the complexity of the scene. Ease of acquisition is another advantage
% of image-based rendering. Rather than requiring complex modeling or measurements, simple photographs are all 
% that is required.
% 
% One common disadvantage of image-based rendering methods is the amount of data storage required. As a result 
% of using a database of images as the representation rather than compact mathematical models storage sizes can be
% very large. Some techniques, such as the PTM method just described, avoid some of the data storage by 
% approximating the images with a simple but powerful model. Another common limitation of image-based 
% methods arises out of sampling issues. It can be difficult to capture high-frequency effects accurately without 
% extremely dense (and time consuming) sampling. Editing can also be problematic for these methods, depending 
% on the representation.
% Despite these limitations image-based rendering and relighting can be very useful for games because of their 
% significant advantages. The recent great leaps in capabilities of graphics hardware have made many more image 
% based techniques useful for a wide range of games and other applications.