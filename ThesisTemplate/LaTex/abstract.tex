
3D textures are often described by parametric functions for each pixel, that
models the variation in its appearance with respect to varying lighting
direction. However, parametric models such as Polynomial Texture Maps (PTMs)
tend to smoothen the changes in appearance. We propose a technique to
effectively model natural material surfaces and their interactions with changing
light conditions. We show that the direct and global components of the image
have different nature, and when modeled separately, leads to a more accurate and
compact model of the 3D surface texture. Direct component is mainly affected by
structural properties of the surface and is therefore deals with phenomena like
shadows and specularity, which are sharply varying functions. The global
component is used to model overall luminance and color values, a smoothly
varying function. For a given lighting position, both components are computed
separately and combined to render a new image. This method models sharp shadows
and specularities, while preserving the structural relief and surface color.
Thus rendered image have enhanced photorealism as compared to images rendered by
existing single pixel models such as PTMs.